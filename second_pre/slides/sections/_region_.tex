\message{ !name(introduction.tex)}
\message{ !name(introduction.tex) !offset(-2) }
%%% Local Variables:
%%% mode: latex
%%% TeX-master: t
%%% End:
\subsection{}
\begin{frame}
\frametitle{Motivation}


\only<1>{


}

\only<2>{


}


\end{frame}
\begin{frame}
  \frametitle{Definition}

\end{frame}

\begin{frame}
  \frametitle{Food Tour Example}
\only<1>{
}
\only<1>{
}
\only<1>{
}
\only<1>{
}
\end{frame}


\begin{frame}
  \frametitle{Generalization}
  \begin{block}{Difficulty of Cuts}
    Minimum cut can be considered as a subset of $k-cut$ where $k$ is
    a fixed number 2.
    \begin{itemize}
    \item $k-cut$ is $NP-Complete$ problem if $k$ is part of the
      input.
    \item Minimum cut is polynomial time calculable.
    \end{itemize}
  \end{block}

!!!!!!!!!!!! we can actually add pictures here for explaination of cut
the NP stuff from wikipedia. Yeah wikipedia bitch!

!!! Also I think what i wrote here is a little dumb though. Please fix
it if you guys have something else.
\end{frame}

\message{ !name(introduction.tex) !offset(-61) }
