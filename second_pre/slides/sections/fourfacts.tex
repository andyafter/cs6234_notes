\begin{frame}
  \frametitle{Sum of Node Degree}
  
  \textbf{Fact 1}
  %TODO We could reformulate this and replace the long if sentence by the simple statement.
  If $degree(u)$ denotes the number of edges touching node $u$, then
  \[
    \sum_{u\in V} degree(u) &= 2|E|
  \]
  because every edge contributes exactly once to the degree of exactly two nodes.
\end{frame}

\begin{frame}
  \frametitle{Average Graph Degree}
  
  \textbf{Fact 2}
  
  If there are $n$ nodes, then the average degree of a node is 2|E|/n.
  
  \[
    E(degree(X)) = \sum_{u\in V} Pr(X=u)\cdot degree(u) = \frac{1}{n} \sum_u degree(u) = \frac{2|E|}{n}
  \]
\end{frame}
